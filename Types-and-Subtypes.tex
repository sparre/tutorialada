\section{Types and Subtypes}

In the examples so far you have seen the built-in types |Integer| and |Character|. Ada also has
a type |Float| for floating point values and a type |Boolean| for true/false values. These types
are similar to their counterparts in other programming languages. However, it may surprise you
to learn that, with the possible exception of |Boolean|, these built-in types are often not used
directly. Instead Ada encourages you to define types that reflect your problem domain and then
use your types throughout your program. There is nothing new about this idea; many languages
support some mechanism for creating user defined types. However, in Ada it is possible to create
even user defined integer, floating point, and character types---a feature many languages do not
support.

To explore this idea further, I will start by introducing Ada's \newterm{strong typing}. There
is no precise definition of this term, but for our purposes we will say that a strongly typed
language is one which does very few, if any, implicit type conversions. Unlike C, for example,
Ada will not convert integers to floating point values or vice-versa without explicit
instruction from the programmer. The example procedure in Listing~\ref{lst:strong-typing}
illustrates the effect.

\begin{figure}[tbhp]
\begin{lstlisting}[
  frame=single,
  xleftmargin=0in,
  caption={Vowel Counting Program},
  label=lst:strong-typing]
procedure Strong_Typing_Example is
   I : Integer;
   F : Float;
begin
   I := 1;    -- Okay.
   I := 1.0;  -- Error. Can't assign a Float to an Integer.
   F := 1;    -- Error. Can't assign an Integer to a Float.
   F := 1.0;  -- Okay.
   F := I;    -- Error.
   I := F;    -- Error.
   F := Float(I);    -- Okay. Explicit conversion.
   I := Integer(F);  -- Okay. Explicit conversion.
end Strong_Typing_Example;
\end{lstlisting}
\end{figure}

If you are not used to strong typing you may find it excessively pedantic and annoying. However,
it exists for a reason. Experience has shown that many bugs first manifest themselves as
confusion about types. If you find yourself mixing types in your program, it may mean that you
have a logical error. Does it make sense to put a value representing aircraft velocity into a
variable that holds a passenger count? Strong typing can catch errors like this, and thus it
promotes software reliability.

By creating a new type for each logically distinct set of values in your program, you enable the
compiler to find logical errors that it might otherwise miss. For example, suppose you are
working on a program that manipulates a two dimensional array of graphical pixels. Instead of
using |Integer| to represent pixel coordinates you might define separate types as follows:

\begin{lstlisting}
type X_Coordinate_Type is new Integer;
type Y_Coordinate_Type is new Integer;
\end{lstlisting}

Here the two types |X_Coordinate_Type| and |Y_Coordinate_Type| are distinct types with no
implicit conversion from one to the other, even though fundamentally both are integer types. Yet
because they are distinct, assigning a Y coordinate value to a variable intended to hold an X
coordinate is now a compile-time error rather than a mistake to be found, if you are lucky,
during testing. For example suppose you had the variables:

\begin{lstlisting}
Base_Length : X_Coordinate_Type;
Side_Length : Y_Coordinate_Type;
\end{lstlisting}

\noindent Now an assignment such as:

\begin{lstlisting}
Side_Length := Base_Length;
\end{lstlisting}

\noindent is caught by the compiler as a type mismatch error. If you actually desire to do this
assignment anyway because, perhaps, you are talking about a square figure, you can use an
explicit type conversion:

\begin{lstlisting}
Side_Length := Y_Coordinate_Type(Base_Length);
\end{lstlisting}

You might feel that distinguishing between coordinate values in two dimensions is overly
restrictive. Yet it may still make sense to define a separate |Coordinate_Type| to distinguish
values that are pixel coordinates from other kinds of integers in your program such as, for
example, line counters, TCP/IP port numbers, user ID numbers, and so forth. In a language like C
where a single type ``int'' might be used to hold all the values I mentioned, the compiler is
powerless to detect illogical mixing of those values across what are really totally different
conceptual domains.

In C++ one could create a class representing each logical concept, and then use overloaded
operators to make instances of that class appear integer-like. However, that is a very
heavy-handed approach to solve what should be a simple problem. Thus, it is not commonly
done.\footnote{If you are a C++ person, as an exercise try writing a C++ template that allows
  you to ``easily'' define distinct, integer-like types. Compare your result with the Ada
  facility described here.}

Ada also allows you to constrain the range of allowed values when you define a new scalar type.
Consider the pixel coordinate example again, and suppose that the drawing area is limited to
2048 pixels in width and 1024 pixels in height. These limitations can be made known to Ada's
type system using type definitions such as:

\begin{lstlisting}
type X_Coordinate_Type is range 0 .. (2048 - 1);
type Y_Coordinate_Type is range 0 .. (1024 - 1);
\end{lstlisting}

Here the (still distinct) types are defined as constrained ranges on |Integer| with the bounds
on |X_Coordinate_Type| running from 0 to 2047 inclusive. When defining new types like this Ada
requires that the ranges be \newterm{static expressions} that can be computed by the compiler.
Thus, an expression such as |2048 - 1| is acceptable but an expression involving a variable is
not.

Whenever a value is assigned to a variable the compiler inserts runtime checks, if necessary, to
verify that the assigned value is in the range of that variable's type. If a range check fails,
the |Constraint_Error| exception is raised. Notice that, unlike the type checking, range checks
are done at runtime. For example:

\begin{lstlisting}
Side_Length := Y_Coordinate_Type(Base_Length); -- Might raise Constraint_Error.
Base_Length := X_Coordinate_Type(Side_Length); -- No Constraint_Error possible.
Side_Length := 1024;  -- Will raise Constraint_Error at runtime.
Side_Length := 1023;  -- No Constraint_Error possible.
\end{lstlisting}

The first assignment above might raise |Constraint_Error| because the value in |Base_Length|
could be out of range for type |Y_Coordinate_Type|. However, the compiler can't know for sure
(in general) without running your program, and so the check is done at runtime. However, the
compiler can be sure that the second assignment above will not raise |Constraint_Error| since
every possible value of |Y_Coordinate_Type| is legitimate for |X_Coordinate_Type|. Thus the
compiler is encouraged, although not required, to ``optimize away'' the runtime check.

The third assignment above clearly tries to put an out of range value into |Side_Length| but the
program will compile anyway. Failed range checks are uniformly handled at runtime; the Ada
language does not require compilers to detect any of them at compile time, no matter how obvious
they may be. That said, a reasonable compiler will notice the obvious range violation in this
example and produce a warning message about it.

Defining your own types also helps you write portable programs. The only integer type compilers
must support is |Integer|. The number of bits used by type |Integer| is implementation defined
and varies from compiler to compiler. Most compilers support additional, non-standard integer
types such as |Long_Integer| and |Long_Long_Integer|. However, the names of these additional
types, and even their existence, varies from compiler to compiler. Rather than deal with this
complexity directly you can just specify the range of values you desire and the compiler will
select the most appropriate underlying type. For example

\begin{lstlisting}
type Block_Counter_Type is range 0 .. 1_000_000_000;
\end{lstlisting}

Variables of type |Block_Counter_Type| may be represented as |Integer| or |Long_Integer| or some
other type as appropriate. In any case they will be constrained to only hold values between zero
and one billion inclusive. If the compiler does not have a built-in integer type with a large
enough range it will produce an error message. If the compiler accepts your type definition you
will not get any surprises. Notice also how Ada allows you to embed underscores in large numbers
to improve their readability.

Ada also allows you to define \newterm{modular types}. These types are unsigned and have
``wrap-around'' semantics. Incrementing beyond the end of an ordinary type causes an exception,
but incrementing beyond the end of a modular type wraps around to zero. In addition the
operators |not|, |and|, |or|, and |xor| can be used on modular types to do bitwise manipulation.
The following listing demonstrates this:

\begin{lstlisting}
type Offset_Type is mod 2**12;  -- Two to the 12th power.
-- The range of Offset_Type is 0 .. 2**12 - 1.
...
Offset : Offset_Type;
...
Offset := Offset and 16#3FF#;   -- Bitwise AND with a hex mask.
\end{lstlisting}

\noindent Here a modular type |Offset_Type| is introduced to hold 12 bit offsets. The variable
|Offset| is masked so that only the lower 10 bits are retained. The snippet above also
demonstrates Ada's exponentiation operator and how numbers in bases other than 10 can be
written. Because Ada was designed for use in embedded systems, its support for low level bit
manipulation is good.

\subsection{Enumeration Types}

In many cases you want to define a type that has only a small number of allowed values and you
want to name those values abstractly. For example, the definition

\begin{lstlisting}
type Day_Type is (Sun, Mon, Tue, Wed, Thu, Fri, Sat);
\end{lstlisting}

\noindent introduces |Day_Type| as an \newterm{enumeration type}. Variables of type |Day_Type|
can only take on the values listed in the type definition. Those values are called
\newterm{enumerators}. This is useful for writing your program in abstract terms that are easy
to understand. Because the compiler treats enumeration types as distinct from integer types (and
from each other) you will not be allowed to mix unrelated concepts without compile time errors.

Enumeration types are useful for modeling real world entities that have a limited number of
distinct states. They can also be used as more descriptive alternatives to the built-in type
|Boolean|. For example:

\begin{lstlisting}
type Color_Type is (Red, Yellow, Green);
type State_Type is (Initializing, Waiting, Processing, Stopping);

-- These types are alternatives to Boolean's True/False.
type Switch_Setting is (Off, On);
type Control_Rod_Setting is (Down, Up);
type System_Mode is (Sleeping, Active);
\end{lstlisting}

Actually, the built-in type |Boolean| is an enumeration type; it behaves exactly as if it was
defined as:

\begin{lstlisting}
type Boolean is (False, True);
\end{lstlisting}

\noindent Even the built-in type |Character| is an enumeration type where the enumerators are
the character literals.

\subsection{Discrete Types}

The integer types and enumeration types are \newterm{discrete types} because they each represent
only a finite set of (ordered) values. All discrete types share some common properties and it is
enlightening to describe those properties in a uniform way.

Ada makes extensive use of \newterm{attributes}. You reference an attribute of a type using an
apostrophe (or ``tick'') immediately following the type name. For example, |Day_Type'First|
represents the first value in the |Day_Type| enumeration and |Integer'First| represents the
first (smallest) allowed value of |Integer|. There is also a |Last| attribute to access the last
allowed value. Some attributes behave like functions. For example |Day_Type'Succ(Sun)| returns
|Mon|, the successor of |Sun| in the |Day_Type| enumeration. There is also a |Pred| attribute
for finding the predecessor value.

In addition there is a |Pos| attribute that returns the integer position of a particular
enumerator and a |Val| attribute that does the reverse. For example |Day_Type'Pos(Mon)| is one
(the position values start at zero) and |Day_Type'Val(1)| is |Mon|. Ada defines many attributes
and we will see others later in this tutorial.

Ranges of discrete types can be defined in a uniform way and used, for example, as cases in
|case| statements and ranges of |for| loops. The following listing shows a contrived example:

\begin{lstlisting}
for Day in Sun .. Fri loop
  case Day in
    when Sun        => -- etc...
    when Mon | Tue  => -- etc...
    when Wed .. Fri => -- etc...
  end case;
end loop;
\end{lstlisting}

\noindent Notice that despite Ada's full coverage rules it is not necessary to specify a case
for |Sat|. This is because the compiler can see |Day| is limited to |Sun .. Fri| and thus will
never take on the value |Sat|. Recall that the loop index variable of a |for| loop can not be
modified inside the loop so there is no possibility of the programmer setting |Day| to |Sat|
before the |case| statement is reached.

\subsection{Subtypes}

As we have seen, when a new type is defined using |type|, the compiler regards it as distinct
from all other types. This is strong typing. However, sometimes it is better to define a
constraint on an existing type rather than introduce an entirely new type. There are several
different kinds of constraints one might define depending on the type being constrained. In this
section I will talk about range constraints on discrete types. As an example, consider the
definition:

\begin{lstlisting}
subtype Weekday is Day_Type range Mon .. Fri;
\end{lstlisting}

\noindent This introduces a \newterm{subtype} named |Weekday| that only contains the values
|Mon| through |Fri|. It is important to understand that a subtype is not a new type, but just a
name for a constrained version of the parent type. The compiler allows variables of a subtype to
be mixed freely with variables of the parent type. Runtime checks are added if necessary to
verify that no value is ever stored in a variable outside the range of its subtype. If an
attempt is made to do so, the |Constraint_Error| exception is raised. The following listing
shows an example:

\begin{lstlisting}
procedure Demonstrate_Subtypes
  (Lower, Upper : in Day_Type; Day : in out Day_Type) is

  subtype Interval is Day_Type range Lower .. Upper;
  X : Interval := Interval'First;
begin
  Day := X;  -- No run time check. Will definitely succeed.
  X := Day;  -- Run time check. Day might be out of range.
End Demonstrate_Subtypes;
\end{lstlisting}

In this example a subtype |Interval| is defined in the procedure's declarative part. The
variable |X| of type |Interval| is given |Interval|'s first value. Mixing |Interval| and
|Day_Type| variables in the later assignment statements is allowed because they are really both
of the same type. Because |Interval| is a subtype of |Day_Type| the assignment of |X| to |Day|
must succeed. Thus the compiler does not need to include any runtime checks on the value of |X|.
However, the value of |Day| might be outside the allowed range of the subtype and so the
compiler will need to insert a runtime check on |Day|'s value before assigning it to |X|. If
that check fails, the |Constraint_Error| exception is raised. Under no circumstances can an out
of bounds value be stored in a variable.

I should point out that in this particular (simplistic) example the compiler's optimizer may be
able to see that the value in |Day| will be in bounds of the subtype since in the assignment
just before it was given a value from the subtype. In that case the compiler is allowed, and
encouraged, to optimize away the run time check that would normally be required. In general Ada
allows any check to be optimized away if the compiler can prove the check will never fail since
doing so will cause no observable change to the program's behavior.

This example also illustrates another important aspect of subtypes: unlike full type, subtypes
can be dynamically defined. Notice that the range on the subtype is taken from the procedure's
parameters. Each time the procedure is called that range might be different. In contrast, as
mentioned previously, the range specified on full type definitions must be static.

A range such as |1 .. 10| is really an abbreviation for the specification of a subtype:

\begin{lstlisting}
Integer range 1 .. 10
\end{lstlisting}

\noindent Thus a |for| loop header such as |for I in 1 .. 10| is really just an abbreviation for
the more specific |for I in Integer range 1 .. 10|. In general the |for| loop specifies a
subtype over which the loop index variable ranges. This is why above it was not necessary to
provide a case for |Sat|. The loop parameter |Day| implicitly declared in the loop header has
the subtype |Day_Type range Sun .. Fri|. The |case| statement contains alternatives for all
possible values in that subtype so the full coverage rules were satisfied.

The Ada environment predefines two important and useful subtypes of |Integer|. Although you
never have to explicitly define these types yourself, the compiler behaves as if the following
two subtype definitions were always directly visible:

\begin{lstlisting}
subtype Natural is Integer range 0 .. Integer'Last;
subtype Positive is Integer range 1 .. Integer'Last;
\end{lstlisting}

\noindent The subtype |Natural| is often useful for counters of various kinds. Since counts
can't be negative the run time checking on |Natural|'s range constraint can help you find
program errors. The subtype |Positive| is useful for cases where zero is a nonsensical value
that should never arise. It is also often used for array indexes as you will see.

\subsection*{Exercises}

\begin{enumerate}
\item Using the definition of |Day_Type| presented earlier would you guess that the following
  works or produces a compiler error: |for Day in Day_Type loop ...|? Write a short program to
  see how the compiler behaves when presented with a loop like this.

\item In the discussion above separate |X_Coordinate_Type| and |Y_Coordinate_Type| definitions
  where introduced. What disadvantage is there to keeping these concepts distinct? Would it be
  better to use a single |Coordinate_Type| for both X and Y coordinates instead? Discuss the
  advantages and disadvantages of both approaches.

\item One question that often comes up in the design of Ada programs is if a certain concept
  should get a distinct type or be defined as a subtype of some other type. What are the
  relative advantages and disadvantages of full types versus subtypes? Think of an example
  (other than the ones discussed here) where two concepts are best represented as separate
  types, and an example where two concepts are best represented by one being a subtype of the
  other.

\item Pick a program you've written in some other language and identify the different ways in
  which the integer type is used. If you were to translate your program to Ada, which of those
  ways would best get new types and which would best be subtypes? Show appropriate type/subtype
  definitions.
\end{enumerate}

