
\chapter{Tutorial}
\label{chapt:tutorial}

Welcome to the Ada programming language! The purpose of this tutorial is to give you an overview
of Ada so that you can start writing Ada programs quickly. This tutorial does not attempt to
cover the entire language. Ada is very large, so complete coverage of all its features would
take many more pages than are contained in this document. However, it is my hope that after
reading this tutorial you will have a good sense of what Ada is like, appreciate some of its
nicer features, and feel interested in learning more
\cite{Barnes2014,McCormick2011,Ben-Ari2009,Burns2007,Dale2007}.

This tutorial assumes you are already familiar with one or more languages in the C family: C,
C++, Java, C\#, or something similar. It is not my intention to teach you how to program. I
assume you already understand concepts such as loops, data types, functions, and so forth.
Instead this tutorial describes how to use these things in Ada. In cases where Ada provides
features that might be unfamiliar to you (such as subtypes, discriminated types, and tasking) I
will discuss those features a bit more comprehensively.

Ada is a powerful language designed to address the following issues:

\begin{itemize}
\item The development of very large programs by multiple, loosely connected teams. The language
  has features to help manage a large number of program components, and to help ensure those
  components are used consistently.

\item The development of long lived programs that spend most of their time in the maintenance
  phase of the software life cycle. The language is designed to promote the readability of
  programs. You may find Ada code to be rather verbose and tedious to write. However, that extra
  work pays off later by making the code clearer and easier to read when bugs must be fixed or
  enhancements written.

\item The development of robust programs where correctness, security, and reliability are
  priorities. The language has features designed to make programming safer and less error prone.
  Some of these features involve extra run time checking and thus entail a performance penalty.
  However, Ada's design is such that the performance penalty is normally not excessive.

\item The development of embedded systems where low level hardware control, multiple concurrent
  tasks, and real time requirements are common. The language has features designed to support
  these things while still retaining as much safety as feasible.
\end{itemize}

\section{Hello, Ada}
\label{sec:hello-ada}

Whenever you begin to study a new language or software development environment you should start
by writing the most trivial program possible in that language or environment. Thus I will begin
this tutorial with a short but complete program that displays the string ``Hello, Ada!'' on the
console.

\label{lst:hello-ada}
\begin{lstlisting}
with Ada.Text_IO;

procedure Hello is
begin
   Ada.Text_IO.Put_Line("Hello, Ada!");
end Hello;
\end{lstlisting}

\noindent The program starts with a \newterm{context clause} consisting of a |with| statement.
The context clause specifies the packages that will be used in this particular compilation unit.
Although the compilation unit above contains just a single procedure, most Ada code is in
packages. The standard library components are in child packages of the package |Ada|. In this
case we will be using facilities in the child package |Ada.Text_IO|.

The main program is the procedure named |Hello|. The precise name used for the main program can
be anything; exactly which procedure is used as the program's entry point is specified when the
program is compiled. The procedure consists of two parts: the part between |is| and |begin| is
called the declarative part. Although empty in this simple case, it is here where you would
declare local variables and other similar things. The part between |begin| and |end| constitute
the executable statements of the procedure. In this case, the only executable statement is a
call to procedure |Put_Line| in package |Ada.Text_IO|. As you can probably guess from its name,
|Put_Line| prints the given string onto the program's standard output device. It also terminates
the output with an appropriate end-of-line marker.

Notice how the name of the procedure is repeated at the end. This is optional, but considered
good practice. In more complex examples the readability is improved by making it clear exactly
what is ending. Notice also the semicolon at the end of the procedure definition. C family
languages do not require a semicolon here so you might accidentally leave it out.

Spelling out the full name |Ada.Text_IO.Put_Line| is rather tedious. If you wish to avoid it you
can include a |use| statement for the |Ada.Text_IO| package in the context clause (or in the
declarative part of |Hello|). Where the |with| statement makes the names in the withed package
\newterm{visible}, the |use| statement makes them \newterm{directly visible}. Such names can
then be used without qualification as shown below:

\begin{lstlisting}
with Ada.Text_IO; use Ada.Text_IO;

procedure Hello is
begin
   Put_Line("Hello, Ada!");
end Hello;
\end{lstlisting}

\noindent Many Ada programmers do this to avoid having to write long package names all the time.
However, indiscriminate use of |use| can make it difficult to understand your program because it
can be hard to tell in which package a particular name has been declared.

Ada is a case insensitive language. Thus identifiers such as |Hello|, |hello|, and |hElLo| all
refer to the same entity. It is traditional in modern Ada source code to use all lower case
letters for reserved words. For all other identifiers, capitalize the first letter of each word
and separate multiple words with an underscore.\footnote{This style is often called ``title
  case.''} The Ada language does not enforce this convention but it is a well established
standard in the Ada community so you should follow it.

Before continuing I should describe how to compile the simple program above. I will assume you
are using the freely available GNAT compiler. This is important because GNAT requires specific
file naming conventions that you must follow. These conventions are not part of the Ada language
and are not necessarily used by other Ada compilers. However, GNAT depends on these conventions
in order to locate the files containing the various compilation units of your program.

The procedure |Hello| should be stored in a file named \filename{hello.adb}. Notice that the
name of the file must be in lower case letters and must agree with the name of the procedure
stored in that file. The \filename{adb} extension stands for ``Ada body.'' This is in contrast
with Ada specification files that are given an extension of \filename{ads}. You will see Ada
specifications when I talk about packages in Section~\ref{sec:packages}. I will describe other
GNAT file naming requirements at that time.

To compile \filename{hello.adb}, open a console (or terminal) window and use the gnatmake
command as follows

\begin{Verbatim}
> gnatmake hello.adb
\end{Verbatim}

\noindent The \command{gnatmake} command will compile the given file and link the resulting
object code into an executable producing, in the above example, \filename{hello.exe} (on
Windows). You can compile Ada programs without \command{gnatmake} by running the compiler and
linker separately. There are sometimes good reasons to do that. However, for the programs you
will write as a beginning Ada programmer, you should get into the habit of using
\command{gnatmake}.

Note that GNAT comes with a graphical Ada programming environment named GPS (GNAT Programming
Studio). GPS is similar to other modern integrated development environments such as Microsoft's
Visual Studio or Eclipse. Feel free to experiment with GPS if you are interested. However, the
use of GPS is outside the scope of this tutorial.

When the compilation has completed successfully you will find that several additional files have
been created. The files \filename{hello.o} and \filename{hello.exe} (on Windows) are the object
file and executable file respectively. The file \filename{hello.ali} is the Ada library
information file. This file is used by GNAT to implement some of the consistency checking
required by the Ada language. It is a plain text file; feel free to look at it. However, you
would normally ignore the \filename{ali} files. If you delete them, they will simply be
regenerated the next time you compile your program.

\subsection*{Exercises}

\begin{enumerate}
\item Enter the trivial ``Hello, Ada'' program on page~\pageref{lst:hello-ada} into your system.
  Compile and run it.

\item Make a minor modification to the trivial program that results in an error. Try compiling
  the program again. Try several different minor errors. This will give you a feeling for the
  kinds of error messages GNAT produces.

\item Experiment with the |use| statement. Try calling |Put_Line| without specifying its
  package, both with and without the |use| statement. Try putting the |use| statement inside the
  declarative part of the procedure. Try putting the |with| statement inside the declarative
  part of the procedure.
\end{enumerate}

\section{Control Structures}
\label{sec:control-structures}

Ada contains all the usual control structures you would expect in a modern language. The program
in Listing~\ref{lst:prime-program} illustrates a few of them, along with several other features.
This program accepts an integer from the user and checks to see if it is a prime number.

\begin{figure}[tbhp]
\begin{lstlisting}[
  frame=single,
  xleftmargin=0in,
  caption={Prime Checking Program},
  label=lst:prime-program]
with Ada.Text_IO;         use Ada.Text_IO;
with Ada.Integer_Text_IO; use Ada.Integer_Text_IO;

procedure Prime is
   Number : Integer;
begin
   Put("Enter an integer: ");
   Get(Number);
   if Number < 2 then
      Put("The value "); Put(Number, 0); Put_Line(" is bad.");
   else
      Put("The value "); Put(Number, 0);
      for I in 2 .. (Number - 1) loop
         if Number rem I = 0 then
            Put_Line(" is not prime.");
            return;
         end if;
      end loop;
      Put_Line(" is prime.");
   end if;
end Prime;
\end{lstlisting}    
\end{figure}

The program declares a local variable named |Number| of type |Integer| in its declarative part.
Notice that the type appears after the name being declared (the opposite of C family languages),
separated from that name with a colon.

Procedure |Get| from package |Ada.Integer_Text_IO| is used to read an integer from the console.
If the value entered is less than two an error message is displayed. Notice that there are two
different |Put| procedures being used: one that outputs a string and another that outputs an
integer. Like C++, Java, and some other modern languages, Ada allows procedure names to be
overloaded distinguishing one procedure from another based on the parameters. In this case the
two different |Put| procedures are also in different packages but that fact is not immediately
evident because of the |use| statements.

If the value entered is two or more, the program uses a |for| loop to see if any value less than
the given number divides into it evenly (that is, produces a zero remainder after division, as
calculated with the |rem| operator). Ada |for| loops scan over the given range assigning each
value in that range to the loop parameter variable (named |I| in this case) one at a time.
Notice that it is not necessary to explicitly declare the loop parameter variable. The compiler
deduces its type based on the type used to define the loop's range. It is also important to
understand that if the start of the range is greater than the end of the range, the result is an
empty range. For example, the range |2 .. 1| contains no members and a loop using that range
won't execute at all. It does not execute starting at two and counting down to one. You need to
use the word |reverse| to get that effect:

\begin{lstlisting}
for I in reverse 1 .. 2 loop
\end{lstlisting}

Notice also that |if| statements require the word |then| and that each |end| is decorated by the
name of the control structure that is ending. Finally notice that a single equal sign is used to
test for equality. There is no ``=='' operator in Ada.

The program in Listing~\ref{lst:vowel-program} counts the vowels in the text at its standard
input. You can provide data to this program either by typing text at the console where you run
it or by using your operating system's I/O redirection operators to connect the program's input
to a text file.

\begin{figure}[tbhp]
\begin{lstlisting}[
  frame=single,
  xleftmargin=0in,
  caption={Vowel Counting Program},
  label=lst:vowel-program]
with Ada.Text_IO;         use Ada.Text_IO;
with Ada.Integer_Text_IO; use Ada.Integer_Text_IO;

procedure Vowels is
   Letter      : Character;
   Vowel_Count : Integer := 0;
   Y_Count     : Integer := 0;
begin
   while not End_Of_File loop
      Get(Letter);
      case Letter is
         when 'A'|'E'|'I'|'O'|'U' |
              'a'|'e'|'i'|'o'|'u' =>
            Vowel_Count := Vowel_Count + 1;
            
         when 'Y'|'y' =>
            Y_Count := Y_Count + 1;
            
         when others =>
            null;
      end case;
   end loop;
   Put("Total number of vowels = "); Put(Vowel_Count); New_Line;
   Put("Total number of Ys = "); Put(Y_Count); New_Line;
end Vowels;
\end{lstlisting}
\end{figure}

This program illustrates |while| loops and |case| structures (similar to C's switch statement).
There are quite a few things to point out about this program. Let's look at them in turn.

\begin{itemize}
\item Variables can be initialized when they are declared. The |:=| symbol is used to give a
  variable its initial value (and also to assign a new value to a variable). Thus like C family
  languages, Ada distinguishes between test for equality (using |=|) and assignment (using
  |:=|). Unlike C family languages you'll never get them mixed up because the compiler will
  catch all incorrect usage as an error.

\item In this program the condition in the |while| loop involves calling the function
  |End_Of_File| in package |Ada.Text_IO|. This function returns |True| if the standard input
  device is in an end-of-file state. Notice that Ada does not require (or even allow) you to use
  an empty parameter list on functions that take no parameters. This means that function
  |End_Of_File| looks like a variable when it is used. This is considered a feature; it means
  that read-only variables can be replaced by parameterless functions without requiring any
  modification of the source code that uses that variable.

\item The program uses the logical operator |not|. There are also operators |and| and |or| that
  can be used in the expected way.

\item The |case| statement branches to the appropriate |when| clause depending on the value in
  the variable |Letter|. The two |when| clauses in this program show multiple alternatives
  separated by vertical bars. If any of the alternatives match, that clause is executed. You can
  also specify ranges in a |when| clause such as, for example, |when 1 .. 10 => etc|.

\item Unlike C family languages there is no ``break'' at the end of each |when| clause. The flow
  of control does \emph{not} fall through to the next |when| clause.

\item Ada has what are called \newterm{full coverage rules}. In a |case| statement you must
  account for every possible value that might occur. Providing |when| clauses for just the
  vowels would be an error since you would not have fully covered all possible characters (the
  type of |Letter| is |Character|). In this program I want to ignore the other characters. To do
  this, I provide a |when others| clause that executes the special |null| statement. This lets
  future readers of my program know that I am ignoring the other characters intentionally and
  it's not just an oversight.

\item |New_Line| is a parameterless procedure in package |Ada.Text_IO| that advances the output
  to the next line. More accurately, |New_Line| is a procedure with a default parameter that can
  be used to specify the number of lines to advance, for example: |New_Line(2)|.
\end{itemize}

\subsection*{Exercises}

\begin{enumerate}
\item The prime number program in Listing~\ref{lst:prime-program} is not very efficient. It can
  be improved by taking advantage of the following observation: If |Number| is not divisible by
  any value less than |I|, then it can't be divisible by any value greater than |Number/I|. Thus
  the upper bound of the loop can be reduced as |I| increases. Modify the program to use this
  observation to improve its performance. Note: You will have to change the |for| loop to a
  |while| loop (try using a |for| loop first and see what happens).

\item Modify the prime number program again so that it loops repeatedly accepting numbers from
  the user until the user types -1. You can program an infinite loop in Ada as shown below. Have
  your program print different error messages for negative numbers (other than -1) than for the
  values zero and one. Use an |if| \ldots\ |elsif| chain (note the spelling of |elsif|). Write a
  version that uses a |case| statement instead of an |if| \ldots\ |elsif| chain. You can use
  |Integer'First| in a range definition to represent the first (smallest) |Integer| value the
  compiler can represent.

  \begin{lstlisting}[escapechar=\@]
  loop
     ...
     exit when @\textit{condition}@
     ...
  end loop;
  \end{lstlisting}

\item \textit{Challenging}. The vowel counting program in Listing~\ref{lst:vowel-program} counts
  `Y' separately because in English `Y' is sometimes a vowel and sometimes not. Modify the
  program so that it adds occurences of `Y' to the vowel counter only when appropriate.
\end{enumerate}

\section{Subprograms}

Unlike C family languages, Ada distinguishes between procedures and functions. Specifically,
functions must return a value and must be called as part of a larger expression. Procedures
never return a value (in the sense that functions do) and must be called in their own
statements. Collectively procedures and functions are called subprograms in situations where
their differences don't matter.

Subprograms can be defined in any declarative part. Thus it is permissible to nest subprogram
definitions inside each other. A nested subprogram has access to the parameters and local
variables in the enclosing subprogram that are defined above the nested subprogram. The scope
rules are largely what you would expect. Below is a variation of the prime number checking
program that introduces a nested function |Is_Prime| that returns |True| if its argument is a
prime number.

\begin{lstlisting}
with Ada.Text_IO;         use Ada.Text_IO;
with Ada.Integer_Text_IO; use Ada.Integer_Text_IO;

procedure Prime2 is
   Number : Integer;

   -- This function returns True if N is prime; False otherwise.
   function Is_Prime(N : Integer) return Boolean is
   begin
      for I in 2 .. (N – 1) loop
         if N mod I = 0 then
            return False;
         end if;
      end loop;
      return True;
   end Is_Prime;

begin
   Put("Enter an integer: ");
   Get(Number);
   if Number < 2 then
      Put("The value "); Put(Number, 0); Put_Line(" is bad.");
   else
      Put("The value "); Put(Number, 0);
      if Is_Prime(Number) then
         Put_Line(" is prime.");
      else
         Put_Line(" is not prime.");
      end if;
   end if;
end Prime2;
\end{lstlisting}

In this case |Is_Prime| has been declared to take a parameter |N| of type |Integer|. It could
have also accessed the local variable |Number| directly. However, it is usually better style to
pass parameters to subprograms in cases where it makes sense. I also wanted to illustrate the
syntax for parameter declarations. Note that because |Is_Prime| is a function a return type must
be specified and it must be called in the context of an expression (in the program above it is
called in the condition of an |if| statement). Procedure declarations are similar except that no
return type is mentioned. Also when a procedure is called it must not be part of an expression,
but rather stand by itself as a single statement. See the calls to procedures |Put| and
|Put_Line| in the examples so far.

The program above also shows the form of a comment. Ada comments start with |--| and run to the
end of the line. As you know, it is good to include comments in your program. However, the
examples in this tutorial do not include many comments in order to save space (and because the
programs are explained in the text anyway).

Procedures also differ from functions in that they support several parameter ``modes.'' The
listing below shows the definition of a procedure that illustrates this. Notice that a
semicolon is used to separate parameter declarations and not a comma as is done in C family
languages.

\begin{lstlisting}
procedure Mode_Example
  (X : in Integer; Y : out Integer; Z : in out Integer) is
begin
  X := 1;      -- Error. Can't modify an in parameter.
  Y := X;      -- Okay. Can read X and modify Y.
  Z := Z + 1;  -- Okay. Can read and write an in out parameter.
end Mode_Example;
\end{lstlisting}

The first parameter declared above has mode |in|. Parameters with this mode are initialized with
the argument provided by the caller, but treated as constants inside the procedure. They can be
read, but not modified. The second parameter has mode |out|. Parameters with this mode are in an
uninitialized state when the procedure begins, but they can be used otherwise as ordinary
variables inside the procedure. Whatever value is assigned to an |out| parameter by the time the
procedure ends is sent back to the calling environment. The third parameter has mode |in out|.
Parameters with this mode are initialized with the argument provided by the caller, and can also
be modified inside the procedure. The changed value is then returned to the caller to replace
the original value. Keep in mind that, unlike in C, modifications to parameters in Ada (when
allowed by the parameter mode) affect the arguments used when the procedure is called.

The mode |in| is the default. In earlier versions of Ada functions parameters could also only
have mode |in| and thus it is common to leave the mode specification off when defining function
parameters. Ada 2012 allows function parameters with any of the three modes. My recommendation,
however, is to always specify the mode when declaring procedure parameters but accept the
default of |in|, except under special circumstances, when declaring function parameters.

Like C++, Ada also allows you to define default values for subprogram parameters. This is
accomplished by initializing the parameter (using the |:=| assignment symbol) when the parameter
is declared. If no argument is provided for that parameter when the subprogram is called the
default value is used instead.

Ada also allows you to call subprograms using named parameter associations. This method allows
you to associate arguments with the parameters in any order, and it can also greatly improve the
readability of your code---particularly if you've chosen good names for the parameters. The
listing below shows how the procedure defined above might be called. Assume that the variables
|Accumulator|, |N|, and |Output| have been previously declared as integers.

\begin{lstlisting}
Mode_Example(Z => Accumulator, X => N+15, Y => Output);
\end{lstlisting}

\noindent Notice that the order in which the arguments are provided is not the same as the order
in which the parameters are declared. When named association is used, the order is no longer
significant. Notice also that you can use any expression as the argument associated with an |in|
parameter. However, |out| and |in out| parameters must be associated with variables and not
arbitrary expressions since putting a value into the result of an expression doesn't really
make sense.

\subsection*{Exercises}

\begin{enumerate}
\item Write a procedure |Count_Primes| that accepts a range of integers and returns the number
  of primes, the smallest prime, and the largest prime in that range. Your implementation should
  use |out| parameters to return its results. It should also make use of the |Is_Prime| function
  defined earlier. Wrap your procedure in a main program that accepts input values from the user
  and outputs the results. Use named parameter association when calling |Count_Primes|.

\item Implement |Count_Primes| from the previous question as a function (or a collection of
  functions) instead. What are the advantages and disadvantages of each approach?
\end{enumerate}

\section{Types and Subtypes}

In the examples so far you have seen the Ada types |Integer|, |Character|, and |Boolean|. Ada
also has a type |Float| for storing floating point values and a type |String|, which is an array
of characters. One important characteristic of Ada that sets it apart from many languages is its
strong typing. There is no precise definition of this term, but for our purposes we will say
that a strongly typed language is one which does no (or very few) automatic type conversions.
Unlike C, Ada will not convert integers to floating point values or vice-versa without explicit
instruction from the programmer. The listing below illustrates this.

\begin{lstlisting}
procedure Strong_Typing_Example is
   I : Integer;
   F : Float;
begin
   I := 1;    -- Okay.
   I := 1.0;  -- Error. Can't assign a Float to an Integer.
   F := 1;    -- Error. Can't assign an Integer to a Float.
   F := 1.0;  -- Okay.
   F := I;    -- Error.
   I := F;    -- Error.
   F := Float(I);    -- Okay. Explicit conversion.
   I := Integer(F);  -- Okay. Explicit conversion.
end Strong_Typing_Example;
\end{lstlisting}

If you are not used to strong typing you may find it excessively pedantic and annoying. However,
it exists for a reason. Experience has shown that many bugs first manifest themselves as
confusion about types. If you find yourself mixing types in your program, it may mean that you
have a logical error. Does it make sense to put a value representing velocity into a variable
that holds a passenger count? Strong typing can catch errors like this and thus it promotes
software reliability.

In fact, Ada allows you to define your own scalar types. By creating a new type for each
logically distinct set of values in your program, you enable the compiler to find logical errors
that it might otherwise miss. For example, consider a procedure |Put_Char| that prints a single
character onto the terminal at a given row and column location. Instead of declaring the row and
column parameters as integers, you might introduce two distinct types for those concepts. The
listing below shows how it could look.

\begin{lstlisting}
type Row_Type is range 1 .. 25;
type Column_Type is range 1 .. 80;

procedure Put_Char(Row : Row_Type; Col : Column_Type; Letter : Character) is
begin
  -- etc
end Put_Char;
\end{lstlisting}

Suppose the main program declared |R| as type |Row_Type| and |C| as type |Column_Type|. Now a
call such as |Put_Char(C, R, 'x')| would be detected by the compiler as an error because the
types used for the first two arguments don't match the types declared for those parameters.
Accidentally switching the row and column arguments in a call like this is an easy error to
make. Ada's strong typing, together with the use of appropriate user defined types, allows the
compiler to catch this error rather than waiting until testing to (maybe) find it.

Strong typing can also find errors in ordinary expressions. For example, consider the assignment
statement |R := 2*C + 1|. The expression on the right hand side has type |Column_Type| but the
variable on the left hand side has type |Row_Type|. It probably doesn't make sense to be
assigning a column value to a row variable and Ada will flag this statement as an error. If it
is really your intention to use this computed column value for a row number, you need to include
an explicit conversion. The result is |R := Row_Type(2*C + 1)|.

You might be wondering what happens in the above example if the result of |2*C + 1| is a value
outside the allowed range specified for type |Row_Type|. It is important to understand that Ada
never allows a value to be assigned to a variable that is outside the set of values for that
variable's type. In this case a run time error would occur (an exception would be raised).

Defining your own types also helps you write portable programs. The only integer type compilers
must support is |Integer|. The number of bits used by type |Integer| is implementation defined
and varies from compiler to compiler. Most compilers support additional, non-standard integer
types such as |Long_Integer| and |Long_Long_Integer|. However, the names of these additional
types, and even their existence, varies from compiler to compiler. Rather than deal with this
complexity directly you can just specify the range of values you desire and the compiler will
select the most appropriate underlying type. For example

\begin{lstlisting}
type Block_Counter_Type is range 0 .. 1_000_000_000;
\end{lstlisting}

Variables of type |Block_Counter_Type| may be represented as |Integer| or |Long_Integer| or some
other type as appropriate. In any case they will be constrained to only hold values between zero
and one billion inclusive. If the compiler does not have a built-in integer type with a large
enough range it will produce an error message. If the compiler accepts your type definition you
will not get any surprises. Notice also how Ada allows you to embed underscores in large numbers
to improve their readability.

Ada also allows you to define modular types. These types are unsigned and have ``wrap-around''
semantics. Incrementing beyond the end of an ordinary type causes an exception, but incrementing
beyond the end of a modular type wraps around to zero. In addition the operators |not|, |and|,
|or|, and |xor| can be used on modular types to do bitwise manipulation. The listing below
demonstrates.

\begin{lstlisting}
type Offset_Type is mod 2**12;  -- Two to the 12th power.
-- The range of Offset_Type is 0 .. 2**12 - 1.
...
Offset : Offset_Type;
...
Offset := Offset and 16#3FF#;   -- Bitwise AND with a hex mask.
\end{lstlisting}

\noindent Here a modular type |Offset_Type| is introduced to hold 12 bit offsets. The variable
|Offset| is masked so that only the lower 10 bits are retained. The snippet above also
demonstrates Ada's exponentiation operator and how numbers in bases other than 10 can be
written. Because Ada was designed for use in embedded systems, its support for low level bit
manipulation is good.

\subsection{Enumeration Types}

In many cases you want to define a type that has only a small number of allowed values and you
want to name those values abstractly. For example, the definition

\begin{lstlisting}
type Day_Type is (Sun, Mon, Tue, Wed, Thu, Fri, Sat);
\end{lstlisting}

\noindent introduces |Day_Type| as an enumeration type. Variables of type |Day_Type| can only
take on the values listed in the type definition. These values are called enumerators. This is
useful for writing your program in abstract terms that are easy to understand. Because the
compiler treats enumeration types as distinct from integer types (and from each other) you will
not be allowed to mix unrelated concepts without compile time errors.

\subsection{Discrete Types}

The integer types and enumeration types are discrete types because they each represent only a
finite set of (ordered) values. All discrete types share some common properties and it is
enlightening to describe those properties in a uniform way.

Ada makes extensive use of attributes. You reference an attribute of a type using an apostrophe
(or ``tick'') immediately following the type name. For example, |Day_Type'First| represents the
first value in the |Day_Type| enumeration and |Integer'First| represents the first (smallest)
allowed value of |Integer|. There is also a |Last| attribute to access the last allowed value.
Some attributes behave like functions. For example |Day_Type'Succ(Sun)| returns |Mon|, the
successor of |Sun| in the |Day_Type| enumeration. There is also a |Pred| attribute for finding
the predecessor value.

In addition there is a |Pos| attribute that returns the integer position of a particular
enumerator and a |Val| attribute that does the reverse. For example |Day_Type'Pos(Mon)| is one
(the position values start at zero) and |Day_Type'Val(1)| is |Mon|. We will see other attributes
later in this tutorial.

Ranges of discrete types can be defined in a uniform way and used, for example, as cases in
|case| statements and ranges of |for| loops. The listing below shows a contrived example.

\begin{lstlisting}
for Day in Sun .. Fri loop
  case Day in
    when Sun        => etc...
    when Mon | Tue  => etc...
    when Wed .. Fri => etc...
  end case;
end loop;
\end{lstlisting}

\noindent Notice that despite Ada's full coverage rules it is not necessary to specify a case
for |Sat|. This is because the compiler can see |Day| is limited to |Sun .. Fri| and thus will
never take on the value |Sat|. Recall that the loop index variable of a |for| loop can not be
modified inside the loop so there is no possibility of the programmer setting |Day| to |Sat|
before the |case| statement is reached.

\subsection{Subtypes}

When a new type is defined using |type|, the compiler regards it as distinct from all other
types. This is strong typing. However, sometimes it is better to define a constraint on an
existing type rather than introduce an entirely new type. There are several different kinds of
constraints one might define depending on the type being constrained. In this section I will
talk about range constraints on discrete types. As an example, the definition

\begin{lstlisting}
subtype Weekday is Day_Type range Mon .. Fri;
\end{lstlisting}

\noindent introduces a subtype named |Weekday| that only contains the values |Mon| through
|Fri|. It is important to understand that a subtype is not a new type, but just a name for a
constrained version of the parent type. The compiler allows variables of a subtype to be mixed
freely with variables of the parent type. Runtime checks are added if necessary to verify that
no value is ever stored in a variable with a subtype that is outside the range of the subtype.
If an attempt is made to do so, the |Constraint_Error| exception is raised. The listing below
shows an example

\begin{lstlisting}
procedure Demonstrate_Subtypes
  (Lower, Upper : in Day_Type; Day : in out Day_Type) is

  subtype Interval is Day_Type range Lower .. Upper;
  X : Interval := Interval'First;
begin
  Day := X;  -- No run time check. Will definitely succeed.
  X := Day;  -- Run time check. Day might be out of range.
End Demonstrate_Subtypes;
\end{lstlisting}

In this example a subtype |Interval| is defined in the procedure's declarative part. The
variable |X| of type |Interval| is given |Interval|'s first value. Mixing |Interval| and
|Day_Type| variables in the later assignment statements is allowed because they are really both
of the same type. Because |Interval| is a subtype of |Day_Type| the assignment of |X| to |Day|
must succeed. Thus the compiler does not need to include any runtime checks on the value of |X|.
However, the value of |Day| might be outside the allowed range of the subtype and so the
compiler will need to insert a run time check on |Day|'s value before assigning it to |X|. If
that check fails, the |Constraint_Error| exception is raised. Under no circumstances can an out
of bounds value be stored in a variable.

I should point out that in this particular (simplistic) example the compiler's optimizer may be
able to see that the value in |Day| will be in bounds of the subtype since in the assignment
just before it was given a value from the subtype. In that case the compiler is allowed and
encouraged to ``optimize away'' the run time check that would normally be required.

This example also illustrates another important aspect of subtypes: they can be dynamically
defined. Notice that the range on the subtype is taken from the procedure's parameters. Each
time the procedure is called that range might be different. In contrast the range specified on
full type definitions must be static (known to the compiler).

A range such as |1 .. 10| is really an abbreviation for the specification of a subtype:

\begin{lstlisting}
Integer range 1 .. 10
\end{lstlisting}

\noindent Thus a |for| loop header such as |for I in 1 .. 10| is really just an abbreviation for
the more specific |for I in Integer range 1 .. 10|. In general the |for| loop specifies a
subtype over which the loop index variable ranges. This is why above it was not necessary to
provide a case for |Sat|. The loop parameter |Day| implicitly declared in the loop header has
the subtype |Day_Type range Sun .. Fri|. The |case| statement contains alternatives for all
possible values in that subtype so the full coverage rules were satisfied.

The Ada environment predefines two important and useful subtypes of |Integer|. Although you
never have to explicitly define these types yourself, the compiler behaves as if the following
two subtype definitions were always directly visible.

\begin{lstlisting}
subtype Natural is Integer range 0 .. Integer'Last;
subtype Positive is Integer range 1 .. Integer'Last;
\end{lstlisting}

\noindent The subtype |Natural| is often useful for counters of various kinds. Since counts
can't be negative the run time checking on |Natural|'s range constraint can help you find
program errors. The subtype |Positive| is useful for cases where zero is a nonsensical value
that should never arise. It is also often used for array indexes as you will see.

\subsection{Never Use Integer}

In Ada you should avoid using the type |Integer| (or |Float|) directly in your program. Instead
it is better practice to introduce your own types or subtypes to name the scalar types you need.
In addition to documenting your program better, this practice allows the compiler or, in the
case of subtypes the runtime system, to catch many logical errors in your code. Ada is designed
to help you write reliable programs, and its strong typing is a major feature in that regard.
However, if you simply declare all your numeric scalar variables as |Integer| uniformly you are
largely negating any benefits strong typing might provide.

In this tutorial |Integer| is used frequently in the examples. This is largely because the
examples are isolated snippets of code that are not really part of a larger program. Examples
also often use short and meaningless variable names for the same reason. In real programs you
should use good variable names and, in Ada, well considered type definitions that embody as many
aspects of your program's design as possible.

\subsection*{Exercises}

\begin{enumerate}
\item Using the definition of |Day_Type| presented earlier would you guess that the following
  works or produces a compiler error: |for Day in Day_Type loop ...|? Write a short program to
  see how the compiler behaves when presented with a loop like this.

\item One of the examples above introduced separate |Row_Type| and a |Column_Type| definitions.
  What disadvantage is there to keeping these concepts distinct? Would it be better to use a
  single |Screen_Coordinate_Type| for both row and column coordinates instead?
\end{enumerate}

\section{Arrays and Records}

Unlike C family languages, arrays in Ada are first class objects. This means you can assign one
array to another, pass arrays into subprograms and return arrays from functions. Every array has
a type, however, this type does not need to be named. For example:

\begin{lstlisting}
Workspace : array(1 .. 1024) of Character;
\end{lstlisting}

defines |Workspace| to be an array of characters indexed using the integers 1 through 1024. Note
that although |Workspace| has an array type, the name of that type is not specified.

In fact, you can define arrays using any discrete subtype for the index (recall that 1..1024 is
really an abbreviation for an integer subtype specification). It is not possible to access an
array out of bounds for exactly the same reason it's not possible to store an out of bounds
value into a variable. The compiler raises the |Constraint_Error| exception in precisely the
same way.

The following example shows many important features.

\begin{lstlisting}
procedure Example is
  type Day_Type is (Sun, Mon, Tue, Wed, Thu, Fri, Sat);
  Work_Hours : array(Day_Type) of Natural;

  function Adjust_Overtime
    (Day : Day_Type; Hours : Natural) return Natural is
  begin
    -- Not shown.
  end Adjust_Overtime;

begin
  Work_Hours := (0, 8, 8, 8, 8, 0);
  for Day in Day_Type loop
    Work_Hours(Day) := Adjust_Overtime(Day, Work_Hours(Day));
  end loop;
end Example;
\end{lstlisting}

\begin{enumerate}

\item Arrays are accessed using parenthesis and not square brackets as in C family languages.
  Thus array access has a syntax similar to that of a function call. An array of constant values
  and can be replaced with a function later without clients needing to be edited.

\item Any discrete subtype, including enumeration subtypes, can be used for an array index. In
  the preceeding example the array |Work_Hours| has elements |Work_Hours(Sun)| and so forth.

\item The nested function |Adjust_Overtime| uses a type defined in the enclosing procedure. This
  is perfectly acceptable. The visibility of all names is handled uniformly.

\item It is possible to use \newterm{array aggregates} in expressions where an array is
  expected. |Work_Hours| is assigned a value that specifies every element of the array in one
  step. The compiler deduces the type of the aggregate from its context.

\item Notice that the same subtype is used for both the loop index variable and the array index.
  This is a common situation and it means the compiler can optimize out the run time checks
  normally required when accessing an array element. Since the value of |Day| can't possibly be
  outside the range of allowed array indexes, there is no need to check that |Work_Hours(Day)|
  is in bounds.

\end{enumerate}

In some cases it is appropriate to give a name to a particular array type. This is desirable if
you plan on declaring many arrays that are intended to have the same type. When the type
definition is in only one place, updating it later is easier. Also you can't assign one array to
another unless they have the same type and in that case the type has to be named. The following
example illustrates.

\begin{lstlisting}
procedure Example is
  type Buffer_Type is array(0..1023) of Character;
  B1 : Buffer_Type;
  B2 : Buffer_Type;
  B3 : array(0..1023) of Character;
begin
  B1 := B2;  -- Fine. Entire array assigned.
  B1 := B3;  -- Error. Types don't match. B3 has anonymous type.
end Example;
\end{lstlisting}

Because subtypes are defined dynamically the size and bounds of an array can also be defined
dynamically. For example a declaration such as |A : array(1..N) of Natural| is allowed even if
the value of |N| is not known at compile time (for example, it could be a subprogram parameter).
This feature means that many places where dynamic allocation is necessary in, for example, C++
can be handled in Ada without the use of explicit memory allocators.

\subsection{Unconstrained Array Types}

One problem with the standard definition of Pascal is that it is not possible to write a
procedure that takes an array of unknown size. At first glance it appears that Ada would also
have that problem. Arrays of different sizes have different types and strong typing will prevent
them from being mixed. This would be a serious limitation. To get around this problem Ada has
the concept of unconstrained array types. Such a type does not specify the bounds on the array,
allowing arrays with different bounds to be in the same type. However, the bounds must be
specified when an actual array object is declared so the compiler knows how much memory to
allocate for the object. The following example illustrates.

\begin{lstlisting}
procedure Unconstrained_Array_Example is
  type Buffer_Type is array(Integer range <>) of Character;
  B1 : Buffer_Type;  -- Error. Must specify array bounds.
  B2 : Buffer_Type( 0..15);  -- Okay.
  B3 : Buffer_Type( 0..15);
  B4 : Buffer_Type(16..31);  -- Fine.
  B5 : Buffer_Type( 0..63);  -- No problem.

  procedure Process_Buffer(Buffer : in Buffer_Type) is
  begin
    for I in Buffer'Range loop
      -- Do something with Buffer(I)
    end loop;
  end Process_Buffer;

begin
  B2 := B3;  -- Fine. Types match and bounds compatible.
  B2 := B4;  -- Fine! Types match and lengths identical.
  B2 := B5;  -- Constraint_Error. Lengths don't match.

  Process_Buffer(B2);  -- Fine.
  Process_Buffer(B4);  -- Fine.
  Process_Buffer(B5);  -- Fine.
end Unconstrained_Array_Example;
\end{lstlisting}

Again there are several points to make about this example.

\begin{enumerate}

\item The symbol |<>| (pronounced “box”) is intended to be a placeholder for information that
  will be filled in later. In this case it specifies that the index bounds on the array type
  |Buffer_Type| is unconstrained.

\item It is illegal to declare a variable with an unconstrained type without providing
  constraints. This is why the declaration of |B1| is an error. However, the missing array
  bounds are specified in the other declarations. Notice that it is not necessary for array
  bounds to start at zero or one. Any particular value is fine as long as it is part of the
  index type mentioned in the array declaration.

\item It is fine to declare a subprogram, such as |Process_Buffer|, taking an unconstrained
  array type as an parameter. However, such a subprogram can't safely access the elements of the
  given array using specific index values like 1 or 2 because those values might not be legal
  for the array. Instead one needs to use special array attributes. For example, |Buffer'First|
  is the first index value that is valid for array |Buffer|, and similarly for |Buffer'Last|.
  The attribute |Range| is shorthand for |Buffer'First..Buffer'Last| and is quite useful in for
  loops as the example illustrates.

\item You might find it surprising that the assignment |B2 := B4| is legal since the array
  bounds do not match. However, the two arrays have the same length so corresponding elements of
  the arrays are assigned. This is called sliding semantics (because you can imagine sliding one
  array over until the bounds do match).

\item All the calls to |Process_Buffer| are fine. Inside the procedure |Buffer'Range| adapts
  itself to whatever array it is given and, provided the code is written with this in mind, the
  procedure works for arrays of any size.

\end{enumerate}

\subsection{Records}

In C family languages composite objects containing components with different types are called
structures. In Ada (and many other languages) they are called records. All record types must
have a name and thus must be declared before any objects of that type can be created. An example
follows.

\begin{lstlisting}
procedure Example is
  type Date is
    record
      Day   : Integer range 1 .. 31;
      Month : Integer range 1 .. 12;
      Year  : Natural;
    end record;

  Today    : Date := (Day => 1, Month => 7, Year => 2007);
  Tomorrow : Date;
begin
  Tomorrow := Today;
  Tomorrow.Day := Tomorrow.Day + 1;
end Example;
\end{lstlisting}

This procedure defines a type |Date| as a collection of three named components. Notice that the
|Day| and |Month| components are defined as subtypes of |Integer| without the bother of naming
the subtypes involved (anonymous subtypes are used). |Today| is declared to be a |Date| and
initialized with a \newterm{record aggregate}. In this case named association is used to make it
clear which component gets which initial value, however positional association is also legal.

Records can be assigned as entire units and their components accessed using the dot operator.
Notice that if |Tomorrow.Day + 1| creates a value that is outside the range |1 .. 31|, a
|Constraint_Error| exception will be raised, as usual when that out of bounds value is assigned
to |Tomorrow.Day|. Obviously a more sophisticated procedure would check for this and adjust the
month as necessary (as well as deal with the fact that not all months are 31 days long).

\section{Packages}
\label{sec:packages}

A package is a named container into which you can place subprograms, type definitions, and other
entities that are related to each other. Packages can even contain other packages. In Ada
packages are the primary way to organize and manage the various parts of a program.

Packages have two parts: a \newterm{specification} that declares the entities that are visible
to the rest of the program, and a \newterm{body} that contains the implementation of those
entities. The body can also contain private entities that can only be used inside the package.
In most cases the specification is stored in a separate file from the body and can even be
compiled separately. This makes the separation between a package's interface and implementation
more rigorous in Ada than in many languages, encouraging the programmer to separate the process
of software design from construction. This formal separation makes it easier for different
groups of people to work on the design and construction of a program since those groups would be
working with different sets of source files. At least this is the ideal situation. Alas, package
specifications can have a private section, discussed later, that is technically part of the
package's implementation. Thus, in some cases at least, interface and implementation end up in
the same file after all.

As an example, consider a simple package for displaying text on a character mode terminal. The
following example shows what the specification might look like. When using the GNAT compiler,
package specifications are stored in files with a \filename{.ads} extension using the same base
name as the name of the package. In this case, the file would be \filename{screen.ads}.

\begin{lstlisting}
package Screen is

   type Row_Type is range 1..25;
   type Column_Type is range 1..80;
   type Color_Type is (Black, Red, Green, Blue, White);

   procedure Set_Cursor_Position
      (Row : in Row_Type; Column : in Column_Type);
   procedure Get_Cursor_Position
      (Row : out Row_Type; Column : out Column_Type);


   procedure Set_Text_Color(Color : in Color_Type);
   procedure Get_Text_Color(Color : out Color_Type);

   procedure Print(Text : String);

end Screen;
\end{lstlisting}

There is no executable code in the specification; it only contains declarations. The example
shows a few types being declared and then some procedures that make use of those types. Packages
can also contain functions.

Once the specification has been written code that makes use of the package can be compiled. Such
a compilation unit contains a statement such as |with Screen| in its context clause. It can
then, for example, refer to the type |Screen.Row_Type| and call procedure |Screen.Print| as you
might expect. Note that a package specification can contain a context clause of its own if it
needs, for example, types defined in some other package.

The package body, stored in a \filename{.adb} file, contains the implementation of the various
subprograms declared in the specification. The Ada compiler checks for consistency between the
specification and the body. You must implement all subprograms declared in the specification and
you must even use exactly the same names for the parameters to those subprograms. However,
the body can define types and subprograms that are not declared in the specification. Such
entities can only be used inside the package body; they are not visible to the users of the
package. The following example shows an abbreviated version of \filename{screen.adb}.

\begin{lstlisting}
package body Screen is

   -- Not declared in the spec. This is for internal use only.
   procedure Helper is
   begin
     -- Implementation not shown.
   end Helper;

   -- All subprograms declared in spec must be implemented.
   procedure Set_Cursor_Position
      (Row : in Row_Type; Column : in Column_Type) is
   begin
     -- Implementation not shown.
   end Set_Cursor_Position;

   -- etc...

end Screen;
\end{lstlisting}

The package body does not need to include a with clause for its own specification, however it
can with other packages as necessary in order to gain access to the resources provided by those
packages.

When building a large program the package specifications can all be written first, before any
real coding starts. The specifications are thus the final output of the design phase of the
software development life cycle. These specifications can be compiled separately to verify that
they are free from syntax errors and that they are internally consistent. The compiler can
verify that all the necessary dependencies are explicitly declared via appropriate |with|
statements, and that all types are used in a manner that is consistent with their declarations.

Once the specifications are in place, the bodies can be written. Because a package body only
depends on the specifications of the packages it uses, the bodies can all be written in
parallel, by different people, without fear of inconsistency creeping into the program.
Programmers could be forbidden to modify any specification files (for example by using
appropriate access control features on a file server or version control system), requiring any
such modifications to first be cleared by a designer. Surprisingly few programming languages
enjoy this level of rigor in the management of code, but Ada was designed for large software
projects and so its features are strong in this area.

\subsection{Package Initialization}

Packages sometimes contain internal variables that need to be initialized in some complex way
before the package can be used. The package author could provide a procedure that does this
initialization, but then there is a risk that it won't get called when it should. Instead, a
package can define some initialization code of arbitrary complexity by introducing an executable
section in the package body. The following example shows how this can be done.

\begin{lstlisting}
with Other;
pragma Elaborate_All(Other);

package body Example is

   Counter : Integer;  -- Internal variable.

   procedure Some_Operation is
   begin
      -- Implementation not shown.
   end Some_Operation;

begin
  Counter := Other.Lookup_Initial("database.vtc.vsc.edu");

end Example;
\end{lstlisting}

In this example, an internal variable |Counter| is initialized by calling a function in a
supporting package |Other|. This initialization is done when the package body is elaborated.
Notice that it is necessary in this example for the body of package |Other| to be elaborated
first so that its initialization statements (if any) execute before the elaboration of the body
of package |Example|. If that were not the case, then function |Lookup_Initial| might not work
properly. To prevent undefined things from happening, an Ada program will instead raise the
exception |Program_Error| at run time if packages get elaborated in an inappropriate order.

To address this potential problem (since you don't really want the exception either) a pragma is
included in the context clause of package |Example|'s body. Pragmas are commands to the compiler
that control the way the program is built. The Ada standard defines several pragmas and
implementations are allowed to define others. In this case the use of pragma |Elaborate_All|
forces the bodies of package |Other| and any packages that package |Other| uses to be elaborated
before the body of package |Example|. Note that the |with| clauses will automatically cause the
specifications of the withed packages to be elaborated first, but not necessarily the bodies. In
this case it is essential to control the order in which the package bodies are elaborated. Hence
the pragma is necessary.

Notice that Ada does not provide any facilities for automatically cleaning up a package with the
program terminates. If a package has shutdown requirements a procedure must be defined for this
purpose, and arrangements must be made to call that procedure at an appropriate time. In
embedded systems, one of the major target application areas for Ada, programs are often never
end. Thus the asymmetric handling of package initialization and cleanup is reasonable in that
context. Other programming languages do provide ``module destructors'' of some kind to deal with
this matter more uniformly.

\subsection{Child Packages}

Being able to put code into various packages is useful, but in a large program the number of
packages might also be large. To cope with this Ada, like many languages, allows its packages to
be organized into a hierarchy of packages, child packages, grand child packages, and so forth.
This allows an entire library to be contained in a single package and yet still allows the
various components of the library to be organized into different packages (or package
hierarchies) as well. This is an essential facility for any language that targets the
construction of large programs.

As an example, consider a hypothetical data compression library. At the top level we might
declare a package |Compress| to contain the entire library. Package |Compress| itself might be
empty with a specification that contains no declarations at all (and no body). Alternatively one
might include a few library-wide type definitions or helper subprograms in the top level
package.

One might then define some child packages of |Compress|. Suppose that |Compress.Algo| contains
the compression algorithms and |Compress.Utility| contains utility subprograms that are used by
the rest of the library but that are not directly related to data compression. Further child
packages of |Compress.Algo| might be defined for each compression algorithm supported by the
library. For example, |Compress.Algo.LZW| might provide types and subprograms related to the LZW
compression algorithm and |Compress.Algo.Huffman| might provide types and subprograms related to
the Huffman encoding compression algorithm. The following example shows a procedure that wishes
to use some of the facilities of this hypothetical library.

\begin{lstlisting}
with Compress.Algo.LZW;

procedure Hello is
  Compressor : Compress.Algo.LZW.LZW_Engine;
begin
  Compress.Algo.LZW.Process(Compressor, "Compress Me!");
end Hello;
\end{lstlisting}

In this example I assume the LZW package provides a type |LZW_Engine| that contains all the
information needed to keep track of the compression algorithm as it works (presumably a record
of some kind). I also assume that the package provides a procedure |Process| that updates the
compressed data using the given string as input.

Clearly a package hierarchy with many levels of child packages will produce very long names for
the entities contained in those deeply nested packages. This can be awkward, but Ada provides
two ways to deal with that. You have already met one way: the |use| statement. By including use
|Compress.Algo.LZW| in the context clause, the contents of that package are made directly
visible and the long prefixes can be deleted. However, Ada also allows you to rename a package
to something shorter and more convenient. The following example shows how it could look.

\begin{lstlisting}
with Compress.Algo.LZW;

procedure Hello is
  package Squeeze renames Compress.Algo;
  Compressor : Squeeze.LZW.LZW_Engine;
begin
  Squeeze.LZW.Process(Compressor “Compress Me!”);
end Hello;
\end{lstlisting}

The package |Squeeze| is not a real package, but rather just a shorter, more convenient name for
an existing package. This allows you to reduce the length of long prefixes without eliminating
them entirely. Although the Ada community encourages the use of long, descriptive names in Ada
programs, names that are local to a single procedure can sensibly be fairly short since they
have limited scope. Using the renaming facility of the language you can introduce abbreviated
names for packages (or other kinds of entities) with limited scope without compromising the
overall readability of the program.

Notice that the optimal names to use for entities in a package will depend on how the package
itself is used. In the example above the package |LZW| provides a type |LZW_Engine|. In cases
(such as shown) where fully qualified names are used the ``LZW'' in the name |LZW_Engine| is
redundant and distracting. It might make sense to just name the type |Engine| yielding a fully
qualified name of |Compress.Algo.LZW.Engine|. On the other hand, if use statements are used the
name |Engine| by itself is rather ambiguous (what kind of engine is that?). In that case it
makes more sense to keep the name |LZW_Engine|. Different programmers have different ideas about
how much name qualification is appropriate and it leads to differences in the way names are
selected.

The problem is that the programmer who write a package is often different from the programmer
who uses it and so incompatible naming styles can arise. Ada's renaming facility gives the using
programmer a certain degree of control over the names that actually appear in his or her code,
despite the names selected by the author of a library. The following example shows a more
radical approach to renaming.

\begin{lstlisting}
with Compress.Algo.LZW;

procedure Hello is
  subtype LZW_Type is Compress.Algo.LZW.LZW_Engine;
  procedure Comp(Engine : LZW_Type; Data : String)
       renames Compress.Algo.LZW.Process;
  Compressor : LZW_Type;
begin
  Comp(Compressor "Compress Me!");
end Hello;
\end{lstlisting}

Notice that while packages and subprograms (and also objects) can be renamed, types can not be
renamed in this way. Instead you can introduce a subtype without any constraints as a way of
effectively renaming a type.

In this simple example, the renaming declarations bulk up the code more than they save. However,
in a longer and more complex situation they can be useful. On the other hand, you should use
renaming declarations cautiously. People reading your code may be confused by the new names if
it is not very clear what they represent.

\section{Abstract Data Types}

Often it is desirable to hide the internal structure of a type from its users. This allows the
designer of that type to change its internal structure later without affecting the code that
uses that type. Ada supports this concept by way of private types. A type can be declared as
private in the visible part of a package specification. Users can then create variables of that
type and use the built in assignment and equality test operations on that type. All other
operations, however, must be provided in the package where the type is defined. Subprograms in
the same package as the type have access to the type's internal representation. The following
example shows a hypothetical package that provides a type representing calendar dates.

\begin{lstlisting}
ackage Example is
   type Date is private;

   function Make(Year, Month, Day : Integer) return Date;
   function Get_Year(Today : Date) return Integer;
   -- etc...
   function Day_Difference(Future, Past : Date) return Integer;
   -- etc...

private
   type Date is
      record
         Y : Integer;
         M : Integer;
         D : Integer;
      end record;
end Example;
\end{lstlisting}

Note that in a more realistic version of this package, one might introduce different types for
year, month, and day values in order to prevent programs from accidentally mixing up those
concepts. However, in the interest of brevity I do not show that in this example. The type
|Date| is declared private in the visible part of the specification. It is then later fully
defined as a record in the private part.

Theoretically the full view of a type should not be in the specification at all; it is part of
the package's implementation and thus should only be in the package body. However, Ada requires
that private types be fully defined in package specifications as a concession to limited
compiler technology. When the compiler translates code that declares variables of the private
type, it needs to know how much memory to set aside for such variables. Thus while the
programmer is forbidden to make use of the private type's internal structure, the compiler needs
to do so. Note that some programming languages go further than Ada in this regard and move all
such private information out of the interface definition. However, there is generally an
execution performance penalty involved in doing this. In Ada it is possible to simulate such a
feature using the so called ``pimpl idiom.'' This method involves access types and controlled
types and so is not described further here.

Note also that the full view of a private type does not need to be a record type. Although
records are common in this situation, private types can be implemented as arrays or even just
simple scalars like |Integer| or |Float|.

Private types do allow assignment and equality tests. However, in some cases it is desirable to
disallow those things. Ada makes this possible using limited private types. A declaration such
as:

\begin{lstlisting}
type Date is limited private;
\end{lstlisting}

tells the compiler to disallow built in assignment and equality tests for the specified type.
This does not mean assignment is impossible; it only means that the package author must now
provide a procedure to do assignment---if support for assignment is desired. Presumably that
procedure would take whatever steps were necessary to make the assignment work properly.
Calendar dates are not good candidates to make into a limited type. The component-wise
assignment of one date to another is exactly the correct way to assign dates. Thus the built in
record assignment operation is fine. Similar comments apply to the built in test for equality
operation. However, types that represent resources outside the computer system (for example a
network connection) can't reasonably be assigned. They should be declared limited private
instead.

\section{Strings}

\emph{To be written\ldots}

\section{Exceptions}

Like many modern programming languages, Ada allows subprograms to report error conditions using
exceptions. When an error condition is detected an exception is raised. The exception propagates
to the dynamically nearest handler for that exception. Once the handler has executed the
exception is said to have been handled and execution continues after the handler. Exceptions are
not resumable. This model is largely the same as that used by C++ although the terminology is
different. However, unlike C++, Ada exceptions are not ordinary objects with ordinary types. In
fact, Ada exceptions don't have a type at all and exist outside the language's type system.

Exceptions are typically defined in a package specification. They are then raised in the package
body under appropriate conditions and handled by the package users. The following example shows
part of a specification for a |Big_Number| package that supports operations on arbitrary
precision integers.

\begin{lstlisting}
package Big_Number is
  type Number_Type is private;

  Divide_By_Zero : exception;

  function "+"(Left, Right : Number_Type) return Number_Type;
  function "-"(Left, Right : Number_Type) return Number_Type;
  function "*"(Left, Right : Number_Type) return Number_Type;
  function "/"(Left, Right : Number_Type) return Number_Type;

private
  -- Not shown.
end Big_Number;
\end{lstlisting}

Notice that Ada, like C++, allows operators to be overloaded. In Ada this is done by defining
functions with names given by the operator names in quotation marks. This package also defines
an exception that will be raised when one attempts to use |Big_Number."/"| with a zero divisor.

The implementation of the division operation might look, in part, as follows:

\begin{lstlisting}
function "/"(Left, Right : Number_Type) return Number_Type is
begin
  -- Assume there is a suitable overloaded “=” operator.
  if Right = 0 then
    raise Divide_By_Zero;
  end if;

  -- Proceed knowing that the divisor is not zero.
end "/";
\end{lstlisting}

As you might guess, the |raise| statement aborts execution of the function immediately. A
procedure that wishes to use |Big_Number| might try to handle this exception in some useful way.
The following example shows the syntax:

\begin{lstlisting}
procedure Example is
  use type Big_Number.Number_Type;
  X, Y, Z : Big_Number.Number_Type;
begin
  -- Fill Y and Z with interesting values.
  X : = Y / Z;
exception
  when Big_Number.Divide_By_Zero =>
    Put_Line("Big Number division by zero!");
  when others =>
    Put_Line("Unexpected exception!");
end Example;
\end{lstlisting}

The block delimited by |begin| and |end| can contain a header |exception|. After that header a
series of |when| clauses specify which exceptions the block is prepared to handle. The special
clause |when others| is optional and is used for all exceptions that are otherwise not
mentioned.

If the normal statements in the block execute without exception, control continues after the
block (skipping over the exception clauses). In the previous example the procedure returns at
that point. If an exception occurs in the block, that exception is matched against those
mentioned in the when clauses. If a match is found the corresponding statements are executed and
then control continues after the block. If no match is found, the block is aborted and the next
dynamically enclosing block is searched for a handler instead. If you are familiar with C++ or
Java exceptions none of this should be surprising.

The previous example also shows an interesting detail related to operator overloading. The
example assumes that there is a context clause of |with Big_Number| (not shown) but no |use|
statement. Thus the division operator is properly named |Big_Number."/"|. Unfortunately it can't
be called using the infix operator notation with that name. There are several ways to get around
this. One could include a |use Big_Number| in the context clause or in the procedure's
declarative part. However that would also make all the other names in package |Big_Number|
directly visible and that might not be desired. An alternative is to introduce a local function
named |"/"| that is just a renaming (essentially an alias) of the function in the other package.
As we have seen Ada allows such renaming declarations in many situations, but in this case,
every operator function would need a corresponding renaming and that could be tedious.

Instead the example shows a more elegant approach. The |use type| declaration tells the compiler
that the primitive operations of the named type should be made directly visible. I will discuss
primitive operations in more detail in the section on object oriented programming. In this case,
the operator overloads that are declared in package |Big_Number| are primitive operations. This
method allows all the operator overloads to be made directly visible with one easy statement,
and yet does not make every name in the package directly visible.

The Ada language specifies four predefined exceptions. These exceptions are declared in package
|Standard| (recall that package |Standard| is effectively built into the compiler) and thus
directly visible at all times. The four predefined exceptions with their use are described as
follows:

\begin{itemize}

\item |Constraint_Error|. Raised whenever a constraint is violated. This includes going outside
  the bounds of a subtype (or equivalently the allowed range of an array index) as well as
  various other situations.

\item |Program_Error|. Raised when certain ill formed programs that can't be detected by the
  compiler are executed. For example, if a function ends without executing a return statement,
  |Program_Error| is raised.

\item |Storage_Error|. Raised when the program is out of memory. This can occur during dynamic
  memory allocation, or be due to a lack of stack space when invoking a subprogram. This can
  also occur during the elaboration of a declarative part (for example if the dynamic bounds
  on an array are too large).

\item |Tasking_Error|. Raised in connection with certain tasking problems. It is outside the
  scope of this tutorial to cover these details.

\end{itemize}

Most of the time the predefined exceptions are raised automatically by the program under the
appropriate conditions. It is legal for you to explicitly raise them if you choose, but it is
recommended that you define your own exceptions for your code. Note that the Ada standard
library defines some additional exceptions as well. Those exceptions are not really predefined
because, unlike the four above, they are not built into the language itself.

\section{Discriminated Types}

\emph{To be written\ldots}

\section{Generics}

Statically checked strongly typed languages like Ada force subprogram argument types and
parameter types to match at compile time. This is awkward in situations where the same sequence
of instructions could be applied to several different types. To satisfy the compile time type
checking requirements, it is necessary to duplicate those instructions for each type being used.
To get around this problem, Ada allows you to define generic subprograms and generic packages
where the types used are parameters to the generic unit. Whenever a new version of the generic
unit is needed, it is \newterm{instantiated} by the compiler for specific values of the type
parameters.

Ada generics are very similar in concept and purpose to C++ templates. Like C++ templates,
generics are handled entirely at compile time; each instantiation generates code that is
specialized for the types used to instantiate it. This is different than generics in Java which
are resolved at run time and that use a common code base for all instances. However, Ada
generics and C++ templates also differ in some important ways. First Ada requires the programmer
to explicitly instantiate a generic unit using special syntax. In C++ instantiations are done
implicitly. Just mentioning the instance causes it to come into existence. In addition, Ada
generics do not support explicit specialization or partial specialization, two features of C++
templates thare used in many advanced template libraries.

There are pros and cons to Ada's approach as compared to C++'s method. With Ada, it is very
clear where the instantiation occurs (since it is shown explicitly in the source code). However,
C++'s method enables advanced programming techniques (template meta-programming) that Ada can't
easily replicate.

The other major difference between Ada generics and C++ templates is that Ada allows the
programmer to specify the necessary properties of the types used to instantiate a generic unit.
In contrast in C++ one only says that the parameter is a type. If a type is used to instantiate
a template that doesn't make sense, the compiler only realizes that when it is actually doing
the instantiation. The resulting error messages are highly cryptic and misleading. In contrast
the Ada compiler can check before it begins the instantiation process if the given types are
acceptable. If they are not, it can provide a much clearer and more specific error message.

To illustrate how Ada generics look, consider the following example that shows the specification
of a generic package containing a number of sorting procedures.

\begin{lstlisting}
generic
  type Element_Type is private;
  with function "<"(Left, Right : Element_Type) return Boolean;
package Sorters is
  type Element_Array is array(Natural range <>) of Element_Type;

  procedure Quick_Sort(Sequence : in out Element_Array);
  procedure Heap_Sort(Sequence : in out Element_Array);
  procedure Bubble_Sort(Sequence : in out Element_Array);
end Sorters;
\end{lstlisting}

Notice that the specification has a generic header that defines the parameters to this generic
unit. The first such parameter is a type that will be called |Element_Type| in the scope of the
generic package. It is declared as a private type to indicate that the package will only (by
default) be able to assign and compare for equality objects of that type. It is thus possible to
instantiate this package for private types, but non-private types like |Integer|, arrays, and
records also have the necessary abilities and can be used. However, a limited type (for example,
a type declared as |limited private| in some other package) can not be used.

This generic package also requires its user to provide a function that can compare two
|Element_Type| objects. That function is called |"<"| in the scope of the package, but it could
be called anything by the user. It must, however, have the same profile (the same number and
type of parameters).

Inside the implementation of the generic package, |Element_Type| can be used as a private type
except that it is also permitted to use |"<"| to compare two |Element_Type| objects. No other
operations on |Element_Type| objects are allowed to guarantee the package will instantiate
correctly with any type the user is allowed to use.

The following example shows how this package might be used:

\begin{lstlisting}
with Sorters;

procedure Example is
  package Integer_Sorters is
    new Sorters(Element_Type => Integer, "<" => Standard."<");

  Data : Integer_Sorters.Element_Array;
begin
  -- Fill Data with interesting information.

  Integer_Sorters.Quick_Sort(Data);
end Example;
\end{lstlisting}

Notice the first line in the declarative region that instantiates the generic unit. In the
example, named parameter association is used to bind the generic unit's arguments to its
parameters. The strange looking construction |"<" => Standard."<"| is an association between the
generic parameter |"<"| (the comparison function) and the operator $<$ that applies to integers
(declared in package |Standard|). It would be more typical for the right side of this
association (and even the left side as well) to be normally named functions.

The name |Integer_Sorters| is given to the specific instance created. That name can then be used
like the name of any other package. In fact, it is legal (and common) to follow a generic
instantiation with a |use| statement to make the contents of the newly instantiated package
directly visible.

Ada also allows procedures and functions to be generic using an analogous syntax.

This example is fairly simple in that the generic type parameter has very limited abilities. It
is also possible to specify that the parameter is a discrete type (thus allowing the use of
|'First|, |'Last|, |'Succ| and |'Pred| attributes), an access type, a tagged type, a modular
type, a floating point type, and various other possibilities. In addition, Ada's generic
facility allows generic units to be parameterized on values (so called ``non-type'' parameters
in C++) and variables. I refer you to one of the references for more information on generic
units in Ada. The discussion here is only scratching the surface of the topic.

\section{Access Types}

Ada, like many languages, allows you to create objects that refer to other objects. It also
allows you to create objects that have a lifetime extending beyond that of the scope in which
they are created. Both of these capabilities are important and many programming techniques
depend on them. C and C++ allow the programmer to use arbitrary pointers, implemented as simple
memory addresses, in any way desired. While this is very powerful, it is also very dangerous. C
and C++ programs suffer from many bugs and security vulnerabilities related to the unrestricted
use of pointers in those languages.

In Ada pointer types are called access types. Like C and C++ pointer variables, access variables
can be manipulated separately from the objects to which they point. However, unlike C and C++
pointer types, access types in Ada have a number of restrictions on their use that are designed
to make them safer.

Actually, the history of access types in Ada is quite interesting. In Ada 83 access types were
very limited but also very safe. However, experience with Ada showed that the limitations were
too great. Ada 95 removed some of the limitations while still managing to keep the safety (at
the expense of complicating the language). Yet even after these changes, access types were still
not as flexible as desired. Ada 2005 removed yet more limitations but now requires, in certain
cases, run time checking to be done to ensure safety. In this tutorial I will not describe
these issues in detail. My focus here is on the basic use of access types in Ada.

Access types can be named or anonymous. I will only consider named access types; anonymous
access types have special properties that are outside the scope of this tutorial. For example
the declaration:

\begin{lstlisting}
type Integer_Access is access Integer;
\end{lstlisting}

declares |Integer_Access| as a type suitable for accessing, or pointing at, integers. Notice
that in this tutorial I have used a suffix of ``Type'' when naming a type. In this case,
however, I use a suffix of ``Access'' to emphasize the nature of the access type. This is, of
course, just a convention.

Once the access type has been declared, variables of that type can then be declared in the usual
way.

\begin{lstlisting}
P : Integer_Access;
\end{lstlisting}

Ada automatically initializes access variables to the special value |null| if no other
initialization is given. Thus access variables are either null or they point at some real
object. Indeed, the rules of Ada are such that dangling pointers are not possible. Access
variables can be copied and compared like any other variable. For example if |P1| and |P2| are
access variables than |P1 = P2| is true if they both point at the same object. To refer to the
object pointed at by an access variable you must use the special |.all| operation.

\begin{lstlisting}
P.all := 1;
\end{lstlisting}

The |.all| plays the same role in Ada as the indirection operator (the *) plays in C and C++.
The use of |.all| may seem a little odd in this context, but understand that most of the time
access types are pointers to records. In that case, the components of the record pointed at can
be accessed directly by using the component selection operation on the access variable itself.
This is shown as follows:

\begin{lstlisting}
type Date is
  record
    Day, Month, Year : Integer;
  end record;
type Date_Access is access Date;

D1, D2 : Date_Access;

...

D1.Day := 1;      -- Accesses the Day member of referenced Date.
D1     := D2;     -- Causes D1 to point at same object as D2.
D1.all := D2.all; –- Copies Date objects.
\end{lstlisting}

In this case the |.all| syntax is very natural. It shows that you are accessing all the members
of the referenced object at the same time. It is important to notice that Ada normally
``forwards'' operations applied to the access variable to the referenced object. Thus |D1.Day|
in the example means the |Day| component of the object pointed at by |D1|. Only operations that
are meaningful for the access type itself are not forwarded. Thus |D1 := D2| copies the access
values. To copy the objects pointed at by the access variables one must use the |.all| syntax. I
should also note that syntax such as |D1.all.Day|, while verbose, is also legal. The |.all|
dereferences |D1|. The result is a date record so the selection of the |Day| component is
meaningful.

Access variables that refer to arrays also allow the normal array operations to be forwarded to
the referenced object as the following example illustrates:

\begin{lstlisting}
type Buffer_Type is array(0..1023) of Character;
type Buffer_Access is access Buffer_Type;
B1, B2 : Buffer_Access;

...

B1     := B2;     -- Copies only the access values.
B1.all := B2.all  -- Copies arrays.
B1(0)  := 'X'     -- Forwards index operation.
for I in B1'Range loop    -- Forwards 'Range attribute.
  ...
end loop;
\end{lstlisting}

Because of forwarding, using access types is generally quite convenient in Ada. However, you
must keep in mind that operations that are meaningful for the access types themselves will be
applied directly to the access objects and are not forwarded.

So far we've seen how to declare and use access types. How does one get an access variable to
point at another object in the first place? In Ada 83 there was only one way: using the new
operation. For example:

\begin{lstlisting}
P := new Integer'(0);
\end{lstlisting}

dynamically allocates an integer, initializes that integer to zero, and then assigns the
resulting access value to |P|. Here I assume |P| has an access to integer type. Notice that
argument to new has the form of a qualified expression (the apostrophe is required). Also as
with dynamically allocated objects in other languages, the lifetime of the object created in
this way extends beyond the lifetime of the access variable used to point at it.

\subsection{Garbage Collection?}

In many languages dynamically allocated objects are automatically reclaimed when they can no
longer be accessed. For example, in Ada parlance, when all access variables pointing at an
object go out of scope the object pointed at by those variables can no longer be referenced and
the memory it uses should be made available again. The process of reclaiming such memory is
called \newterm{garbage collection}.

In Ada garbage collection is optional. Implementations are allowed to provide it, but they are
not required to do so. This may seem surprisingly wishy-washy for a language that endeavors to
support reliable and portable programming. The problem is that Ada also endeavors to support low
level embedded systems and real time programming. In such environments garbage collection is
widely considered problematic. It is difficult to meet real time objectives if a complex garbage
collection algorithm might run at any time. Also the space overhead of having a garbage
collector in the run time system might be unacceptable for space constrained embedded devices.
Advances in garbage collection technology and machine capabilities have made some of these
concerns less pressing today than they were when Ada was first designed. However, these matters
are still important. Thus Ada allows, but does not require garbage collection.

This presents an immediate problem for programmers interested in portability. If the
implementation does not collect its garbage and the programmer takes no steps to manually
reclaim allocated objects, the program will leak memory. This is a disaster for long running
programs like servers. Thus for maximum portability one must assume that garbage collection is
not done and take steps accordingly. In fact, most Ada implementations do not provide garbage
collection, so this is a very realistic concern.

The Ada library provides a generic procedure named |Unchecked_Deallocation| that can be used to
manually deallocate a dynamically allocated object. Unfortunately the use of
|Unchecked_Deallocation| can violate important safety properties the language otherwise
provides. In particular, if you deallocate an object while an access variable still points at
it, any further use of that access variable will result in erroneous behavior1. As a service
|Unchecked_Deallocation| will set the access variable you give it to |null| causing future use
of that access variable to result in a well defined exception. However, there might be other
access variables that point at the same object and |Unchecked_Deallocation| can not know about
all of them in general.

If your program never uses |Unchecked_Deallocation| then all access variables are either |null|
or point at a real object; dangling pointers are impossible. However, your program might also
leak memory if the Ada implementation does not provide garbage collection. Thus in most real
programs |Unchecked_Deallocation| is used.

This is an example of where Ada compromises safety for the sake of practical reality. In fact,
there are several other ``Unchecked'' operations in Ada that are used to address certain
practical concerns and yet introduce the possibility of unsafe programs. Since every such
operation starts with the word ``Unchecked'' it is an simple matter to search an Ada program for
occurrences of them. This feature makes reviewing the unchecked operations easier.

\textit{Show an example of Unchecked\_Deallocation\ldots}

\section{Command Line Arguments}

\textit{To be written\ldots}

\section{Object Oriented Programming}

\textit{To be written\ldots}

\section{Tasking}

Many programs can be naturally described as multiple, interacting, concurrent threads of
execution. Programming environments that provide direct support for concurrency are thus very
useful. Such support can be offered by the operating system or by the programming language or
some combination of the two.

The classic way in which operating systems support concurrency is by allowing multiple
independent processes to run simultaneously. This method has the advantage of offering good
isolation between the processes so that if one crashes the others are not necessarily affected.
On the other hand, communication between processes tends to have high overhead. Modern operating
systems also usually allow programs to be written that have multiple threads of control
executing in the same process. This \newterm{thread level concurrency} is harder to program
correctly but has reduced overhead as compared to the older style \newterm{process level
  concurrency}.

In some environments offering thread level concurrency the programmer must invoke subprograms in
a special library to create and synchronize threads. Such an approach requires relatively little
support from the programming language but it tends to be error prone. Another approach is to
build support for thread level concurrency into the programming language itself. This allows the
compiler to take care of the low level details of thread management, freeing the programmer to
focus on other aspects of the program's design. Ada uses this second approach and supports
concurrent programming as a language feature.

The unit of execution in Ada is called a task. The main program executes in a task called the
environment task. You can create additional tasks as appropriate to meet your application's
needs. Like a package each task has a specification and a body. In the simplest case a task
specification only needs to declare that the task exists. The following example illustrates the
basic syntax.

\begin{lstlisting}
with Ada.Text_IO;
with Helper;

procedure Main is

  – Specification of nested task.
  task Nag;

  – Body of nested task.
  task body Nag is
  begin
    for I in 1 .. 100 loop
      Ada.Text_IO.Put_Line("Hello");
      delay 10.0;
    end loop;
  end Nag;

begin
  Helper.Something_Useful;
end Main;
\end{lstlisting}

In this example a task |Nag| is both specified and defined in the declarative part of the main
program. The task simply prints ``Hello'' one hundred times with a 10 second delay between each
line of output. While the task does this important work, the main program simultaneously
executes the useful function of the program.

It is important to understand that the task starts executing automatically as soon as the
enclosing subprogram begins. It is not necessary to explicitly start the task. Furthermore the
enclosing subprogram will not return until the task has completed. Care must be taken to ensure
that the task eventually ends. If the task never ends the enclosing subprogram will never
return.

Because the task is nested inside another program unit it has access to the local variables and
other entities declared in the enclosing unit above the task's definition. This gives a way for
the task to share information with the enclosing unit but beware that sharing data in this
manner is difficult and error prone. As I will describe shortly Ada provides much more robust
ways for tasks to communicate.

The example above shows a task nested inside a procedure. Tasks can also be nested inside
functions or even inside each other. This allows you to create a task to assist in the execution
of any subprogram without anyone outside your subprogram being aware that tasks are involved.
You can also create tasks inside a package body that support the operation of that package. Be
aware, however, that if you do create a task inside a library package body you need to arrange
for that task to eventually end or else the program will never terminate.

\textit{Finish me\ldots}

\section{Container Library}

\textit{To be written\ldots}

\section{Low Level Programming}

\textit{To be written\ldots}

