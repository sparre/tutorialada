\section{Generics}

Statically checked strongly typed languages like Ada force subprogram argument types and
parameter types to match at compile time. This is awkward in situations where the same sequence
of instructions could be applied to several different types. To satisfy the compile time type
checking requirements, it is necessary to duplicate those instructions for each type being used.
To get around this problem, Ada allows you to define generic subprograms and generic packages
where the types used are parameters to the generic unit. Whenever a new version of the generic
unit is needed, it is \newterm{instantiated} by the compiler for specific values of the type
parameters.

Ada generics are very similar in concept and purpose to C++ templates. Like C++ templates,
generics are handled entirely at compile time; each instantiation generates code that is
specialized for the types used to instantiate it. This is different than generics in Java which
are resolved at run time and that use a common code base for all instances. However, Ada
generics and C++ templates also differ in some important ways. First Ada requires the programmer
to explicitly instantiate a generic unit using special syntax. In C++ instantiations are done
implicitly. In addition, Ada generics do not support explicit specialization or partial
specialization, two features of C++ templates that are used in many advanced template libraries.

There are pros and cons to Ada's approach as compared to C++'s method. With Ada, it is very
clear where the instantiation occurs (since it is shown explicitly in the source code). However,
C++'s method enables advanced programming techniques (template meta-programming) that Ada can't
easily replicate.

The other major difference between Ada generics and C++ templates is that Ada allows the
programmer to specify the necessary properties of the types used to instantiate a generic unit.
In contrast in C++ one only says that the parameter is a type. If a type is used to instantiate
a template that doesn't make sense, the compiler only realizes that when it is actually doing
the instantiation. The resulting error messages can be very cryptic. In contrast the Ada
compiler can check before it begins the instantiation if the given types are acceptable. If they
are not, it can provide a much clearer and more specific error message.

To illustrate how Ada generics look, consider the following example that shows the specification
of a generic package containing a number of sorting procedures.

\begin{lstlisting}
generic
  type Element_Type is private;
  with function "<"(Left, Right : Element_Type) return Boolean is <>;
package Sorters is
  type Element_Array is array(Natural range <>) of Element_Type;

  procedure Quick_Sort(Sequence : in out Element_Array);
  procedure Heap_Sort(Sequence : in out Element_Array);
  procedure Bubble_Sort(Sequence : in out Element_Array);
end Sorters;
\end{lstlisting}

Notice that the specification has a generic header that defines the parameters to this generic
unit. The first such parameter is a type that will be called |Element_Type| in the scope of the
generic package. It is declared as a private type to indicate that the package will only (by
default) be able to assign and compare for equality objects of that type. It is thus possible to
instantiate this package for private types, but non-private types like |Integer|, arrays, and
records also have the necessary abilities and can be used. However, a limited type (for example,
a type declared as |limited private| in some other package) can not be used.

This generic package also requires its user to provide a function that can compare two
|Element_Type| objects. That function is called |"<"| in the scope of the package, but it could
be called anything by the user. It must, however, have the same profile (the same number and
type of parameters). The |<>| symbol at the end of the function parameter indicates that the
compiler should automatically fill in the required function if one is available at the
instantiation point. This allows the user to instantiate the package using just one type
argument if the default |"<"| for that type is acceptable.

Inside the implementation of the generic package, |Element_Type| can be used as a private type
except that it is also permitted to use |"<"| to compare two |Element_Type| objects. No other
operations on |Element_Type| objects are allowed to guarantee the package will instantiate
correctly with any type the user is allowed to use.

The following example shows how this package might be used:

\begin{lstlisting}
with Sorters;

procedure Example is
  package Integer_Sorters is
    new Sorters(Element_Type => Integer, "<" => Standard."<");

  Data : Integer_Sorters.Element_Array;
begin
  -- Fill Data with interesting information.

  Integer_Sorters.Quick_Sort(Data);
end Example;
\end{lstlisting}

Notice the first line in the declarative region that instantiates the generic unit. In the
example, named parameter association is used to bind the generic unit's arguments to its
parameters. The strange looking construction |"<" => Standard."<"| is an association between the
generic parameter |"<"| (the comparison function) and the operator $<$ that applies to integers
(declared in package |Standard|). It would be more typical for the right side of this
association (and even the left side as well) to be named functions.

In fact the second parameter of the instantiation is not actually necessary in this case
because, due to the |<>| symbol used in the generic parameter declaration, the compiler will
fill in the proper function automatically. However, it is still possible to provide the function
explicitly in cases where some different function is desired.

The name |Integer_Sorters| is given to the specific instance created. That name can then be used
like the name of any other package. In fact, it is legal (and common) to follow a generic
instantiation with a |use| statement to make the contents of the newly instantiated package
directly visible.

Ada also allows procedures and functions to be generic using an analogous syntax.

This example is fairly simple in that the generic type parameter has very limited abilities. It
is also possible to specify that the parameter is a discrete type (thus allowing the use of
|'First|, |'Last|, |'Succ| and |'Pred| attributes), an access type, a tagged type, a modular
type, a floating point type, and various other possibilities. In addition, Ada's generic
facility allows generic units to be parameterized on values (so called ``non-type'' parameters
in C++) and variables. I refer you to one of the references for more information on generic
units in Ada. The discussion here is only scratching the surface of this large and important
topic.

